\section{Verwendete Tools}\label{sec:kaptiel}
\subsection{Wireshark}
\cite{WS1} Wireshark ist ein mächtiges Tool zur Analyse der Netzwerkkommunikation.
Es werden alle Pakete erfasst die im Netzwerk versendet werden und in einem 
lesbaren Format dargestellt. 
Standardmäßig wird Wireshark im 'promiscous mode' gestartet. Dieser ermöglicht
es, nicht nur Datenverkehr des eigenen Netzwerkadapters zu sehen, sondern alle Pakete
des Netzwerkes. 
Soll eine bestimmte Anwendung analysiert werden, sind viele der empfangenen Pakete
uninteressant. Wireshark stellt ein mächtiges Filtertool zur verfügung. 
Beispielsweise können Pakete mit einem bestimmten Format, einer bestimmten IP-Adresse, 
bestimmten Schlüsselbegriffen etc. angezeigt werden. Zur Unterstützung der Eingabe, werden
zahlreiche Standardfilter angezeigt. 
Es gibt zwei Möglichkeiten mithilfe von Wireshark Pakete zu filtern.
\begin{itemize}
    \item Capture Filter: hier werden die Pakete nur erfasst, wenn sie zu dem eingestellten Filter passen
    \item Display Filter: hier werden alle Pakete erfasst, jedoch nur diese dargestellt, die zu einem Filter passen
\end{itemize}


% \subsection{BurpSuite}
% \subsection{PortSniffer}
% \subsection{Chrome Debugger}
