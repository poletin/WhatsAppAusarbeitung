\section{Fazit}\label{sec:fazit}
Für die Übertragung von Informationen verwendet WhatsApp den neues Stand der Technik. 
Während der Analyse des eigenen WLANs ist deutlich geworden, dass die Pakete von WhatsApp 
die einzigen Pakete waren, die mit TLSv1.3 verschlüsselt wurden. 
Die Ende-zu-Ende-Verschlüsselung der Nachrichten verwendet mit AES und dem Deffie-Hellman Verfahren zum Austausch der SChlüssel
eine sichere Methode für die Datensicherheit. 
WhatsApp behauptet die privaten Schlüssel der Clients nicht zu kennen, diese seien nicht auf den Servern gespeichert, 
sondern lediglich auf den End-Geräten verfügbar. Dies müsste allerdings in einem weiteren 
Projekt analysiert werden. 
Durch die Verschlüsselung und die Verwendung von TLSv1.3 sind bei der Analyse der Pakete nur wenige Informationen 
verfügbar. Zu erkennen sind die Verfahren und Protokolle die WhatsApp verwendet. 
Diese werden allerdings auch von WhatsApp nicht geheimgehalten, denn die Verfahren sind nicht durch ihre Geheimhaltung sicher, sondern durch 
die Geheimhaltung des privaten Schlüssels. WhatsApp hat sich in dem letzten Jahr sehr viel Gedanken bzgl. der Nachrichtenübertragung im Bezug 
auf Sicherheit gemacht und verbessern diese stetig. 