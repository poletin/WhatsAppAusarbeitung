\section{Einleitung}\label{sec:einleitung}
% \todo[inline]{Sinnvolle Einleitung schreiben}
% \blindtext

% Daher bleibt festzustellen,
% dass es für Latex keinen \enquote{\ac{wysiwyg}}-Editor gibt.
% \ac{wysiwyg} ist eine andere Möglichkeit Dokumente zu erstellen.
WhatsApp ersetzt immer mehr alle Bereiche der Kommunikation sei es SMS, Anrufe oder Email.
Doch trotz der Popularität steht WhatsApp immer wieder in der Kritik. Kümmert sich das Unternehmen genug um
den Schutz der Daten? Anfang 2016 führte WhatsApp eine Ende-zu-Ende-Verschlüsselung. Zuvor war 
es möglich, fremde Nachrichten im Klartext abzufangen. Durch die Ende-zu-Ende-Verschlüsselung, ist dies
seit ca. 1 1/2 Jahren nicht mehr so einfach möglich. Doch immer wieder werden Meldungen bekannt gegeben, in denen 
Schwachstellen im Sicherheitssystem von WhatsApp gefunden wurden. WhatsApp arbeitet seither stetig an der Verbesserung
der Sicherheit. So wurde beispielsweise das Unternehmen "Checkpoint" engagiert um durch "penetration testing" Schwachstellen zu finden, die im Anschluss
von WhatsApp geschlossen werden können. 
Um eine eigene Einschätzung der Sicherheit von WhatsApp zu erhalten, ist es notwendig sich zunächst damit auseinander zusetzen
wie die Kommunikation bei WhatsApp funktioniert. In dieser Ausarbeitung soll dieser Grundstein gelegt werden. 
Folgende Fragen werden geklärt: 

\begin{itemize}
    \item Wie funktioniert die Authentifizierung beim Login?
    \item Welche Pakete werden für den versandt von Nachrichten über das lokale Netzwerk versendet?
    \item Welchen Weg nehmen die Pakete bei der Verwendung von WebWhatsApp?
\end{itemize}

